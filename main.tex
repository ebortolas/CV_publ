\documentclass[11pt,a4paper]{article}
\usepackage[utf8]{inputenc}
\usepackage[english]{babel}
\usepackage{amssymb}
\usepackage[inner=2.2cm,outer=2.2cm,top=1.7cm,bottom=1.5cm]{geometry}
\usepackage{tikz}
\usepackage{enumitem}
\setlist[enumerate]{itemsep=10cm}

\usepackage{lettrine}
\usepackage{mathptmx}

%indentazione
\setlength{\parindent}{0pt}
\setlength{\parskip}{0pt}



\usepackage{sectsty,ulem} 
\definecolor{color}{RGB}{145, 0, 37}
\definecolor{linkc}{RGB}{0,200,200}

\subsectionfont{\color{color}  }

%\renewcommand{\labelitemi}{\textperiodcentered}

\usepackage[compact]{titlesec}
\titlespacing{\subsection}{0pt}{7pt}{1pt}
\renewcommand{\labelitemi}{\color{color}\raisebox{2pt}{\tiny $\bullet$}}

\usepackage{marvosym}
%\usepackage{amsfonts}
%\usepackage{makeidx}
%\usepackage[pdftex]{graphicx}
%\usepackage{kpfonts}
%\usepackage{adjustbox}
%\usepackage{scrextend} %footnotes
%\usepackage{float}
%\usepackage{wrapfig}

%\usepackage{lipsum}
%\usepackage{type1cm}
%\usepackage{fancyhdr}
%\setlist{itemsep=1pt}

\setlist[itemize]{%
  topsep=5pt,               % space before start / after end of list
  itemsep=-1pt,               % space between items
  font={\normalfont}, % set the label font
  leftmargin = 3.5mm
}
\setlist[description]{%
  topsep=5pt,               % space before start / after end of list
  itemsep=0pt,               % space between items
  font={\normalfont}, % set the label font
  leftmargin = 5.5mm,
  labelindent=5pt
}
\usepackage[itemsep=-1pt, topsep=5pt, leftmargin = 7mm, labelsep=3pt]{etaremune}

\usepackage{verbatim}

\usepackage[colorlinks=false,%
allbordercolors={linkc},%
pdfborderstyle={/S/U/W 1}]{hyperref}

 
\usepackage{fancyhdr}
\pagestyle{fancy}
\fancyhf{}
\fancyhead[L]{\it Elisa Bortolas \ -- \ Curriculum Vitae }
\fancyhead[R]{\thepage}

\begin{document}

\thispagestyle{plain}

\vspace*{-1.6cm}
\begin{flushright}
{ \it \today \ \ \ \ \ \  }
\end{flushright}


\begin{center}
\begin{Large}
{\color{color} {\bf Elisa Bortolas} } -- Curriculum Vitae
\end{Large}

\vspace{0.17cm}
Center for Theoretical Astrophysics and Cosmology, Institute for Computational Science, \\ University of Z\"urich, Winterthurerstrasse
190, CH-8057 Z\"urich, Switzerland\\
Website: { \href{https://ebortolas.wixsite.com/ebortolas}{https://ebortolas.wixsite.com/ebortolas}}; tel: +41 (0) 44 63 56689; \ \ \  e-mail: elisa.bortolas@uzh.ch
\end{center}

\vspace{0.17cm}

\subsection*{Current Position}

{\bf Postdoctoral research fellow} at the University of Z\"urich, Switzerland. 

\subsection*{Summary of scientific interests}
\begin{itemize}
%Star and gas dynamics in galactic nuclei -- Dynamical evolution of supermassive black hole binaries -- Computational Astrophysics and high performance computing
\item Dynamics and evolution of supermassive black hole binaries 
%Low frequency 
\item Gravitational wave sources
\item Dynamics of galactic nuclei, with a particular focus to our Galactic Centre \item Computational Astrophysics and high performance computing
\end{itemize}

\subsection*{Employment}
\begin{description}
  \item[\normalfont 11/2018 -- now:] Postdoctoral research fellow, Institute for Computational Science,  University of Z\"urich 
\end{description}

\subsection*{Education}
\begin{description}
    \item[2015 – 2019:] {\bf Ph.D. in Astronomy} obtained from the University of Padova (Italy) on March 5$^{\rm th}$, 2019. 
    Thesis: \textit{Dynamics of Single and Binary Black Holes in Galactic Nuclei}\\
    Supervisor: Prof M. Mapelli; co-Supervisor: Dr A. Gualandris. 
    
    \item[2013 – 2015:] {\bf Master degree in Astrophysics and Space Physics} obtained from the University of Milano-Bicocca (Italy) on September 21$^{\rm st}$, 2015; \ mark 110/110~{\bf cum laude}. \\
    Thesis: \textit{ \href{https://drive.google.com/file/d/0Bx7KflRPRjKbZTRkc19yeXdLc1E/view?usp=sharing}{Dynamics of Massive Black Hole Binaries in Galactic Merger Remnants}}.\\
    Internal supervisor: Prof Massimo Dotti; external supervisor: Dr Alessia Gualandris.
    
    \item[2010 – 2013:] {\bf Bachelor degree in Physics} obtained from the University of Milano-Bicocca (Italy) on October 21$^{\rm st}$, 2013; mark 110/110~{\bf cum laude}.\\
    Thesis: \textit{Analysis of a sample of mega-maser hosting galaxies}.\\ Supervisor: Prof Massimo Dotti; co-supervisor: Prof Giuseppe Gavazzi.

\end{description}

\subsection*{Prizes, awards and fellowships}

\begin{description}

    \item[2019:] \textbf{Livio Gratton prize for the best Italian PhD thesis in Astrophysics} in 2017 -- 2018;\\ { INAF press release: \href{https://www.media.inaf.it/2019/07/18/il-premio-gratton-a-una-tesi-sui-buchi-neri/}{\small \normalfont https://www.media.inaf.it/2019/07/18/il-premio-gratton-a-una-tesi-sui-buchi-neri/}}

    \item[2018:] \textbf{Honourable mention, GWIC-Braccini PhD Thesis Prize}; international yearly prize for the best PhD thesis in the field of gravitational waves {\small (\href{https://gwic.ligo.org/thesisprize/2018/}{https://gwic.ligo.org/thesisprize/2018/})}. 

    \item[2017:] {\textbf{\href{https://www.arcetri.astro.it/ricerca/premio-stefano-magini}{Stefano Magini Prize}}} {\bf for the best Italian Master thesis in Astrophysics} in 2015 -- 2016.\\
    { INAF press release: \href{http://www.media.inaf.it/2017/05/29/premio-magini-arcetri/}{ \small www.media.inaf.it/2017/05/29/premio-magini-arcetri/}}

    \item[2016:]  {\textbf{Ing. Gini Fellowship}} to support a period of research abroad for Italian students graduated in STEM

\end{description}

\subsection*{Grants}

\begin{description}
%\setlength\itemsep{-2pt}

    \item[2019:] {\bf Gratton Prize} for the best Italian PhD thesis in Astrophysics 2017 -- 2018,  {\bf 5,000~EUR}

    \item[2017:] {\bf Magini Prize} for the best Italian master thesis in Astrophysics 2015 -- 2016,  {\bf 1,250~EUR}

    \item[2016:]{ \bf Ing. Gini Fellowship} to support a period of study or research abroad,  {\bf 6,400~EUR}

    \item[2015:] {\bf Erasmus+ Traineeship} to support a research period  in a foreign EU institution,  {\bf 3,000~EUR}

\end{description}


\subsection*{Teaching experience}

\begin{description}
%\setlength\itemsep{-1pt}
    \item[2019:] \textbf{Guest lecturer} for the   \textit{Computational Astrophysics Lab course}, University of Z\"urich; I also developed the material on the direct summation $N$-body technique for the students final project.

    \item[2019:] 
    \textbf{Teaching Assistant}, Course \textit{The Universe: Contents, Origin, Evolution and Future}, University of Z\"urich.

    \item[2018:] Granted 24 European University Credits in Psychology and Pedagogy for teaching in secondary school, University of Padova (Italy).
\end{description}

\subsection*{Students Advising}

\begin{description}

    
    \item[\bf PhD students:] R. Souza-Lima (University of Z\"urich, 2019 -- 2020)

    \item[\bf Master students:] Boris Pestoni (University of Z\"urich, 2019), Lorenz Zwick (University of Z\"urich, 2019), Nikolas Wittek (University of Z\"urich, 2019), Jacob Stegmann (University of Z\"urich, 2019)
    
    \item[\bf Bachelor students:]  A. Incatasciato (University of Milano Bicocca, 2015)


\end{description}

\subsection*{Extended research visits}
%\vspace{0.5mm}
\begin{description}
%\setlength\itemsep{-1pt}

    \item[2017:] {  Visiting PhD Student } at the {\bf University of Surrey}, UK, 10 months. 

    \item[2015:] { Erasmus+ Traineeship Programme} (Internship) at the {\bf University of Surrey}, UK, 6 months.  

\end{description}

\subsection*{Invited Colloquia}

\begin{etaremune}
%\setlength\itemsep{-1pt}

    \item {\bf Seminar} at the University of Zurich (Switzerland), January 25$^{\rm th}$, 2019; title: {\it A long journey to gravitational waves: the evolution of supermassive black hole binaries in galaxy merger remnants}

    \item {\bf Seminar} at the University of Innsbruck (Austria), March 12$^{\rm th}$, 2018; title: {\it Supernova Kicks and Dynamics of Compact Remnants in the Galactic Centre}

    \item {\bf Talk} at the University of Zurich (Switzerland), November 27$^{\rm th}$, 2017; title: {\it Destination gravitational waves: a journey of supermassive black hole binaries and their host galaxies}

    \item {\bf Talk} for the Stefano Magini Prize, May 30$^{\rm th}$, 2017 at the INAF – Osservatorio Astronomico di Arcetri (Italy); title: {\it Brownian Motion of Massive Black Hole Binaries in Galaxy Merger Remnants}

\end{etaremune}


\subsection*{Conferences and Workshops}

\begin{etaremune}
%\setlength\itemsep{-1pt}

    \item \textsc{YAGN20 meeting}, June 8--10, 2020 at the Niels Bohr Institutet of Copenhagen (Denmark);  {\bf invited talk} {\it(forthcoming)}.


    \item \textsc{YAGN19 meeting}, September 23--25, 2019 in  Tenerife (Spain); \textbf{summary talk (theory)} and {\bf invited talk}  {\it Evolution of supermassive black hole pairs at $z\sim 6$}.

    \item \textsc{Tal Alexander Meeting 2019,}%meeting focusing on {\it Extreme Mass Ratio Inspirals}
     August 25 -- September 5, 2019 in Al\'ajar (Spain); {\bf invited blackboard presentation} on the dynamics of extreme mass ratio inspirals.

    \item \textsc{1$^{\rm st}$ LISA Astrophysics Working Group Workshop}, December 12--14, 2018 in  Paris (France); 
 {\bf talk} {\it Can supernova kicks trigger EMRIs in the Galactic Centre?}

    \item \textsc{YAGN18 meeting}, October 29--31, 2018 in  Budapest (Hungary); {\bf invited talk} {\it Can supernova kicks trigger EMRIs in the Galactic Centre?}

    \item \textsc{MODEST-18 meeting}, June 25--29, 2018 in Santorini (Greece); {\bf talk} {\it  EMRIs Triggered by Supernova Kicks in the Galactic Center}.

    \item \textsc{GRAvitational-wave Science \& technology Symposium 2018,} March 1 -- 2, 2018  in Padova (Italy);  {\bf talk} {\it Star Cluster Disruption by a Super-massive Black Hole Binary}.


    \item \textsc{Astro-GR@Barcelona 2017,} meeting focusing on {\it Extreme Mass Ratio Inspirals}, October 16 -- 20, 2017 in Barcelona (Spain).

    \item \textsc{ MODEST-17 meeting},  September 18 -- 22, 2017 in Prague (Czech Republic):
\vspace{-10pt}

    \begin{itemize}[leftmargin=5mm]
    \setlength\itemsep{-2pt}
        \item {\bf talk} {\it Star Cluster Disruption by a Supermassive Black Hole Binary}; 
        \item {\bf poster presentation} {\it Supernova Kicks and Dynamics of Compact Remnants in the Galactic Centre}; 
        \item {\bf chair}  of the {\it Galactic Centre / Nuclear Star Clusters} session.
    
    \end{itemize}
    \vspace{-8pt}

    \item \textsc{Lemaitre Workshop}  on Black Holes, Gravitational Waves and Spacetime Singularities, May 9 -- 12, 2017, Vatican Observatory, Albano Laziale (Italy); {\bf talk} {\it Massive Black Hole Binaries and their Hosts: a common journey towards Gravitational Waves}.

    \item \textsc{Cosmic-Lab – MODEST-16 meeting},  April 18 -- 22, 2016 in Bologna (Italy); {\bf poster presentation} {\it Dynamics of Supernova remnants in the Galactic Center}.

    \item \textsc{MODEST-15s meeting}, December 7--11, 2015 in Kobe (Japan);  {\bf talk} {\it Dynamics of Massive Black Hole Binaries in Galactic Merger Remnants}.

\end{etaremune}


\subsection*{Accepted computational proposals}

\begin{description}

    \item[\normalfont 2016:] \textbf{PI} of the proposal\textit{ Evolution of Massive Black Hole Binaries in their host systems} at CINECA, {\bf 99k CPU hours} awarded for $N$-body simulations on the GALILEO IBM cluster.

    \item[ \normalfont  2015:] \textbf{PI} of the proposal \textit{Evolution of Massive Black Hole Binaries} at CINECA, {\bf 99k CPU hours} awarded for $N$-body simulations on the GALILEO IBM cluster.
\end{description}

\subsection*{IT experience}

\begin{itemize}

    \item\textit{Operating Systems}: Linux/Unix, Windows

    \item\textit{Programming Languages}: {\bf C } (advanced), C++ (good), python (good), bash (good)
    \item \textit{Codes}: HiGPUs, HiGPUs-R, phiGRAPE, phiGPU, RAMSES, gasoline2

    \item\textit{Science-related softwares/packages}: ROOT, IRAF, SPLASH; python: Numpy, Scipy, yt, pynbody

    \item\textit{Plotting packages}: gnuplot, Supermongo, SPLASH; python: Matplotlib and Pyplot 
    
    \item\textit{Job Schedulers}: PBS, Slurm
    
    \item\textbf{Experience in GPU computing and parallel computing} (CUDA, OpenCL; openMP, MPI)
    
    \item Experience in code optimization and high performance computing 

\end{itemize}

\subsection*{Memberships}

\begin{itemize}
%\setlength\itemsep{-2pt}

    \item Full member of the {Astrophysics Working Group} of the {\bf LISA Consortium} since 2018; member of the LISA Early Career Scientist Group (LECS) since 2019.

    \item Co-I of the White Book for the scientific cases of  \textbf{MAORY}, the multi-conjugate adaptive optics module of the European Extremely Large Telescope.
    
\end{itemize}


\subsection*{Referee duties}
Reviewer for the {\it Monthly Notices of the Royal Astronomical Society} (MNRAS) since 2018.

\subsection*{Advanced courses and schools (highlights)}

\begin{itemize}
%\setlength\itemsep{-2pt}

    \item \href{http://ihpcss18.it4i.cz/}{\textit{International HPC Summer School 2018,}} {\bf highly competitive admission} (80 participants selected among $\sim$600 applicants); July 8--13, 2018 in Ostrava (Czech Republic)

    %\item Course { {\it Machine Learning for Astronomy}}, held for the PhD School in Astronomy at the University of Padova by Dr M. Pasquato, February 15--22, 2018 
    
    \item 11$^{\rm th}$ TRR33 Winter School in Cosmology, {\it Theory for Observers \& Observations for Theorists}, December 10--16, 2017, Passo del Tonale (Italy); { poster presentation} {\it Supernova Kicks and Dynamics of Compact Remnants in the Galactic Centre}, \textbf{  first prize for the best poster of the school} 
    
    %\item Course {\it Dynamics of stars and black holes in dense stellar systems} held for the PhD School in Astronomy at the University of Padova by Prof M. Mapelli, June 2017
    
    %\item {\it KROME computational school}, September 19--21, 2016 in Arcetri, Florence

    \item CINECA school \textit{{ 25$^{\rm th}$ Summer School on Parallel Computing}}, June 6--17, 2016 in Casalecchio di Reno, Bologna (Italy)

    %\item CINECA course \textit{{ Python for computational science}}, February 9--11, 2016 in Segrate, Milan
    
    %\item Course {\it N-body techniques for astrophysics} held for the PhD School in Astronomy at the University of Padova by Prof M. Mapelli, November 2015.
    
\end{itemize}

\subsection*{Outreach activities}

\begin{itemize}
%\setlength\itemsep{-4pt}

    \item Public outreach seminar at the Auditorium of Varano Borghi (Italy), 2020 {\it(forthcoming)}
    
    \item  Public outreach talk for the Livio Gratton Prize  in Frascati (Italy), 2019;\\
    Title: \textit{Quando due giganti si incontrano: una storia sui buchi neri supermassicci} %(The rendez-vous of giants: a supermassive black hole story)

    \item Outreach seminar at the Padova Planetarium, 2018;\\
    Title: \textit{A caccia di buchi neri con le onde gravitazionali}% (Hunting black holes with gravitational waves)

    \item Member of the committee for the organization of \href{https://pintofscience.com/}{\it Pint of Science} 2018 in Padova (Italy);
    
    \item \href{http://www.venetonight.it/}{\it Notte dei Ricercatori} (The night of researchers) 2017, public outreach events in Padova  (Italy);
    
    %\item[2017:] INAF press release for the Magini Prize 2016: %\href{http://www.media.inaf.it/2017/05/29/premio-magini-arcetri/}{www.media.inaf.it/2017/05/29/premio-magini-arcetri/}
    
    \item Support for the monthly public star gazing events held at the University of Surrey telescope, 2016 -- 2017.

\end{itemize}



\subsection*{Institutional responsibilities and Service}

\begin{description}

    \item[ \normalfont 2019:] Organizer of the Seminars for the Institute of Computational Science, University of Z\"urich
    
    \item[ \normalfont 2019:] Main organizer of a series of  events  for the \textbf{promotion of Diversity in Science}, Institute for Computational Sciences, University of Z\"urich; I was the main speaker in one of these meetings; website: \href{https://padlet.com/eli_bortolas/39e0898pr0uu}{https://padlet.com/eli\_bortolas/39e0898pr0uu}.

    \item[ \normalfont 2013 -- 2015:] Student representative of the Astrophysics and Space Physics master students at the University of Milano-Bicocca.
    
\end{description}


\subsection*{Language knowledge}
Italian (mother tongue), English (fluent), French (basic).%, German (basic).


%\noindent\rule{\textwidth}{1pt}
\pagebreak


\fancyhead[L]{\it Elisa Bortolas \ -- \ Curriculum Vitae and Publications List }

\vspace{-4mm}

\subsection*{\LARGE{Publications  List}}

\vspace{3mm}

\textit{Symbol \# marks publications in collaboration with students I co-advised.}

\vspace{2mm}

\subsection*{ Refereed publications}

\begin{etaremune}

\item {\bf Bortolas, E.} \& Mapelli, M. { \it Can Supernova Kicks trigger EMRIs in the Galactic Centre?}, 2019, MNRAS, 485, 2125 \\ \href{https://ui.adsabs.harvard.edu/abs/2019MNRAS.485.2125B/abstract}{\scriptsize [https://ui.adsabs.harvard.edu/abs/2019MNRAS.485.2125B/abstract]}

\item {\bf Bortolas, E.}, Gualandris, A., Dotti, M., Read, J. I., { \it The influence of Massive Black Hole Binaries on the Morphology of Merger Remnants},  2018, MNRAS 477, 2310 \\ \href{https://ui.adsabs.harvard.edu/abs/2018MNRAS.477.2310B/abstract}{\scriptsize [https://ui.adsabs.harvard.edu/abs/2018MNRAS.477.2310B/abstract]}

\item {\bf Bortolas, E.}, Mapelli, M., Spera, M., { \it Star cluster disruption by a massive black hole binary}, 2018, MNRAS 474,1054 \\ \href{https://ui.adsabs.harvard.edu/abs/2018MNRAS.474.1054B/abstract}{\scriptsize [https://ui.adsabs.harvard.edu/abs/2018MNRAS.474.1054B/abstract]}

\item {\bf Bortolas, E.}, Mapelli, M., Spera, M., {\it Supernova Kicks and Dynamics of Compact Remnants in the Galactic Centre}, 2017, MNRAS 469, 1510 \\ \href{https://ui.adsabs.harvard.edu/abs/2017MNRAS.469.1510B/abstract}{\scriptsize [https://ui.adsabs.harvard.edu/abs/2017MNRAS.469.1510B/abstract]} 

\item Gualandris, A., Read, J. I., Dehnen, W., {\bf Bortolas, E.}, {\it Collisionless loss-cone refilling: there is no final parsec problem}, 2017, MNRAS, 464, 2301 
\\ \href{https://ui.adsabs.harvard.edu/abs/2017MNRAS.464.2301G/abstract}{\scriptsize [https://ui.adsabs.harvard.edu/abs/2017MNRAS.464.2301G/abstract]}


\item {\bf Bortolas, E.}, Gualandris, A., Dotti, M., Spera, M., Mapelli, M., {\it Brownian motion of massive black hole binaries and the final parsec problem}, 2016, MNRAS 461, 1023 \\ \href{https://ui.adsabs.harvard.edu/abs/2016MNRAS.461.1023B/abstract}{\scriptsize [https://ui.adsabs.harvard.edu/abs/2016MNRAS.461.1023B/abstract]}


\end{etaremune}

\subsection*{ Submitted papers }

\begin{itemize}
\setlength\itemsep{-1pt}
\item  \# Stegmann, J., Capelo, P. R.,  {\bf Bortolas, E.}, Mayer, L., { \it Improved constraints from ultra-faint dwarf galaxies on primordial black holes as dark matter}, submitted to MNRAS \\ \href{https://ui.adsabs.harvard.edu/abs/2019arXiv191004793S/abstract}{\scriptsize [https://ui.adsabs.harvard.edu/abs/2019arXiv191004793S/abstract]}

\item  \# Zwick, L., Capelo, P. R., \textbf{Bortolas, E.}, Mayer, L., Amaro-Seoane, P. { \it Improved gravitational radiation time-scales: significance for LISA and LIGO-Virgo sources}, submitted to MNRAS \\ \href{https://ui.adsabs.harvard.edu/abs/arXiv\%3A1911.06024/abstract}{\scriptsize [https://ui.adsabs.harvard.edu/abs/arXiv\%3A1911.06024/abstract]}
\end{itemize}


\subsection*{Conference Proceedings }



\begin{etaremune}

\item {\bf Bortolas, E.}, Mapelli, M., Spera, M., {\it Star Cluster Disruption by a Supermassive Black Hole Binary}, to appear in {\it Proceedings of Science}  (proceedings of the  GRAvitational-waves Science \& technology Symposium - GRASS2018, 1-2 March 2018, Palazzo Moroni, Padova, Italy) \\ \href{https://ui.adsabs.harvard.edu/abs/2018gwss.confE..30B/abstract}{\scriptsize [https://ui.adsabs.harvard.edu/abs/2018gwss.confE..30B/abstract]}




\item {\bf Bortolas, E.}, Mapelli, M., Spera, M., {\it Dynamics of supernova remnants in the Galactic Centre}, to appear in {\it Cosmic-Lab: Star Clusters as Cosmic Laboratories for Astrophysics, Dynamics and Fundamental Physics}, F.R. Ferraro and B. Lanzoni eds, Mem. SAIt, Vol 87 (proceedings of the Cosmic-Lab conference, 18-22 April 2016, Bologna, Italy)\\ \href{https://ui.adsabs.harvard.edu/abs/2016MmSAI..87..679B/abstract}{\scriptsize [https://ui.adsabs.harvard.edu/abs/2016MmSAI..87..679B/abstract]}



\end{etaremune}

\subsection*{Theses}


\begin{itemize}
   \item {\normalfont November 2018:} {\bf Bortolas, E.}, {\it Dynamics of Single and Binary Black Holes in Galactic Nuclei}, PhD Thesis in Astronomy \href{http://paduaresearch.cab.unipd.it/11825}{\scriptsize [http://paduaresearch.cab.unipd.it/11825]}
  \item {\normalfont September 2015:} {\bf Bortolas, E.}, {\it Dynamics of Single and Binary Black Holes in Galactic Nuclei}, Master Thesis in Astrophysics and Space Physics \href{https://drive.google.com/open?id=0Bx7KflRPRjKbZTRkc19yeXdLc1E}{\scriptsize [https://drive.google.com/open?id=0Bx7KflRPRjKbZTRkc19yeXdLc1E]}
\end{itemize}


%%%%%%%%%%%%%%%%%%%%%%%%%%%%%%%%%%%%%%%%%%%%%%%%%%%%%%%%%%%%%%%%
\begin{comment}
\noindent\rule{\textwidth}{0.1pt}
\subsection{References}
\begin{itemize}
\setlength\itemsep{-2pt}

\item Prof Massimo \textsc{Dotti}; Universit\`a degli Studi di Milano-Bicocca; massimo.dotti@mib.infn.it

\item Dr Alessia \textsc{Gualandris}; University of Surrey; a.gualandris@surrey.ac.uk

\item Prof Michela \textsc{Mapelli}; Leopold Franzens Universit\"at Innsbruck {\it and} INAF -- Osservatorio Astronomico di Padova; michela.mapelli@oapd.inaf.it

\item Prof Justin I. \textsc{Read}, University of Surrey; j.read@surrey.ac.uk
\end{itemize}
\end{comment}
%%%%%%%%%%%%%%%%%%%%%%%%%%%%%%%%%%%%%%%%%%%%%%%%%%%%%%%%%%%%%%%%
\end{document}