
\subsection*{Current Position}

{\bf Postdoctoral research fellow} at the University of Z\"urich, Switzerland. 

\subsection*{Summary of scientific interests}
\begin{itemize}
%Star and gas dynamics in galactic nuclei -- Dynamical evolution of supermassive black hole binaries -- Computational Astrophysics and high performance computing
\item Dynamics and evolution of supermassive black hole binaries 
\item Extreme mass ratio inspirals and tidal disruption events
%Low frequency 
\item Gravitational wave sources
\item Dynamics of galactic nuclei, with a particular focus to our Galactic Centre \item Computational Astrophysics and high performance computing
\end{itemize}

\subsection*{Employment}
\begin{description}
  \item[\normalfont 11/2018 -- now:] Postdoctoral research fellow (Assistant before April 2019), Institute for Computational Science,  University of Z\"urich 
\end{description}

\subsection*{Education}
\begin{description}
    \item[2015 – 2019:] {\bf Ph.D. in Astronomy} obtained from the University of Padova (Italy) on March 5$^{\rm th}$, 2019. 
    Thesis: \textit{Dynamics of Single and Binary Black Holes in Galactic Nuclei}\\
    Supervisor: Prof M. Mapelli; co-Supervisor: Dr A. Gualandris. 
    
    \item[2013 – 2015:] {\bf Master degree in Astrophysics and Space Physics} obtained from the University of Milano-Bicocca (Italy) on September 21$^{\rm st}$, 2015; \ mark 110/110~{\bf cum laude}. \\
    Thesis: \textit{ \href{https://drive.google.com/file/d/0Bx7KflRPRjKbZTRkc19yeXdLc1E/view?usp=sharing}{Dynamics of Massive Black Hole Binaries in Galactic Merger Remnants}}.\\
    Internal supervisor: Prof M. Dotti; external supervisor: Dr A. Gualandris.
    
    \item[2010 – 2013:] {\bf Bachelor degree in Physics} obtained from the University of Milano-Bicocca (Italy) on October 21$^{\rm st}$, 2013; mark 110/110~{\bf cum laude}.\\
    Thesis: \textit{Analysis of a sample of mega-maser hosting galaxies}.\\ Supervisor: Prof M. Dotti; co-supervisor: Prof G. Gavazzi.

\end{description}

\subsection*{Prizes, awards and fellowships}

\begin{description}

    \item[2019:] \textbf{Livio Gratton prize for the best Italian PhD thesis in Astrophysics} in 2017 -- 2018;\\ { INAF press release: \href{https://www.media.inaf.it/2019/07/18/il-premio-gratton-a-una-tesi-sui-buchi-neri/}{\small \normalfont https://www.media.inaf.it/2019/07/18/il-premio-gratton-a-una-tesi-sui-buchi-neri/}}

    \item[2018:] \textbf{Honourable mention, GWIC-Braccini PhD Thesis Prize}; international yearly prize for the best PhD thesis in the field of gravitational waves {\small (\href{https://gwic.ligo.org/thesisprize/2018/}{https://gwic.ligo.org/thesisprize/2018/})}. 

    \item[2017:] {\textbf{\href{https://www.arcetri.astro.it/ricerca/premio-stefano-magini}{Stefano Magini Prize}}} {\bf for the best Italian Master thesis in Astrophysics} in 2015 -- 2016.\\
    { INAF press release: \href{http://www.media.inaf.it/2017/05/29/premio-magini-arcetri/}{ \small www.media.inaf.it/2017/05/29/premio-magini-arcetri/}}

    \item[2016:]  {\textbf{Ing. Gini Fellowship}} to support a period of research abroad for Italian students graduated in STEM

\end{description}

\subsection*{Grants}

\begin{description}
%\setlength\itemsep{-2pt}

    \item[2019:] {\bf Gratton Prize} for the best Italian PhD thesis in Astrophysics 2017 -- 2018,  {\bf 5,000~EUR}

    \item[2017:] {\bf Magini Prize} for the best Italian master thesis in Astrophysics 2015 -- 2016,  {\bf 1,250~EUR}

    \item[2016:]{ \bf Ing. Gini Fellowship} to support a period of study or research abroad,  {\bf 6,400~EUR}

    \item[2015:] {\bf Erasmus+ Traineeship} to support a research period  in a foreign EU institution,  {\bf 3,000~EUR}

\end{description}


\subsection*{Teaching experience}

\begin{description}
%\setlength\itemsep{-1pt}

    \item[2020:] 
    \textbf{Teaching Assistant}, Course \textit{The Universe: Contents, Origin, Evolution and Future}, U. of Z\"urich.
    
    \item[2019:] \textbf{Guest Lecturer} and \textbf{Teaching Assistant},   \textit{Computational Astrophysics Lab course}, U. of Z\"urich. %; I also developed the material on the direct summation $N$-body technique for the students final project.

    \item[2019:] 
    \textbf{Teaching Assistant}, Course \textit{The Universe: Contents, Origin, Evolution and Future}, U. of Z\"urich.

    \item[2018:] Granted 24 European University Credits in Psychology and Pedagogy for teaching in secondary school, U. of Padova (Italy).
\end{description}

\subsection*{Students Mentoring}

\begin{description}

    
    \item[\bf PhD students:] \underline{Lorenz Zwick} (U. of Z\"urich, 2019 -- 2020; main supervisor: L. Mayer), \underline{Rafael Souza-Lima} (U. of Z\"urich, 2019 -- 2020; main supervisor: L. Mayer); \underline{Verónica Vázquez Aceves} (Peking U., 2020; main supervisor: X. Chen)

    \item[\bf Master students:] \underline{Boris Pestoni} (U. of Z\"urich, 2019; main supervisor: L. Mayer), \underline{Lorenz Zwick} (U. of Z\"urich, 2019; main supervisor: L. Mayer), \underline{Nikolas Wittek} (U. of Z\"urich, 2019; main supervisor: L. Mayer), \underline{Jakob Stegmann} (U. of Z\"urich, 2019; main supervisor: L. Mayer), \underline{Ludovica Varisco} (U. of Milano Bicocca, 2020; main supervisor: M. Dotti)
    
    \item[\bf Bachelor students:]  \underline{Andrea Incatasciato} (U. of Milano Bicocca, 2015; main supervisor: M. Dotti)


\end{description}

\subsection*{Extended research visits}
%\vspace{0.5mm}
\begin{description}
%\setlength\itemsep{-1pt}

    \item[2017:] {  Visiting PhD Student } at the {\bf University of Surrey}, UK, 10 months. 

    \item[2015:] { Erasmus+ Traineeship Programme} (Internship) at the {\bf University of Surrey}, UK, 6 months.  

\end{description}

\subsection*{Invited Colloquia}

\begin{etaremune}
%\setlength\itemsep{-1pt}

    \item {\bf Seminar} at the University of Zurich (Switzerland), January 25$^{\rm th}$, 2019; title: {\it A long journey to gravitational waves: the evolution of supermassive black hole binaries in galaxy merger remnants}

    \item {\bf Seminar} at the University of Innsbruck (Austria), March 12$^{\rm th}$, 2018; title: {\it Supernova Kicks and Dynamics of Compact Remnants in the Galactic Centre}

    \item {\bf Talk} at the University of Zurich (Switzerland), November 27$^{\rm th}$, 2017; title: {\it Destination gravitational waves: a journey of supermassive black hole binaries and their host galaxies}

    \item {\bf Talk} for the Stefano Magini Prize, May 30$^{\rm th}$, 2017 at the INAF – Osservatorio Astronomico di Arcetri (Italy); title: {\it Brownian Motion of Massive Black Hole Binaries in Galaxy Merger Remnants}

\end{etaremune}


\subsection*{Conferences and Workshops}

\begin{etaremune}
%\setlength\itemsep{-1pt}

    \item \textsc{YAGN20 meeting}, October 28--30, 2020, Niels Bohr Institutet of Copenhagen (Denmark);  {\bf invited talk}  \textit{Course or blessing? The role of bars in the large-scale pairing of massive black holes}
    \textit{(forthcoming)}.
    
        \item \textsc{LISA Astrophysics Working Group Workshop}, March 9--11, 2018 in  Nijmegen,  {\bf talk} {\it The stochastic pairing of massive black holes in the young Universe}.
    
    \item \textsc{Sexten Workshop}, \textit{Getting ready to descend the slippery slope of multimessenger cosmological black holes data}, February 10--14, 2020 at the Sexten Centre for Astrophysics (Italy);  {\bf invited talk} \textit{The stochastic pairing of massive black hole binaries in the young Universe}.

    \item \textsc{YAGN19 meeting}, September 23--25, 2019 in  Tenerife (Spain); \textbf{summary talk (theory)} and {\bf invited talk}  {\it Evolution of supermassive black hole pairs at $z\sim 6$}.

    \item \textsc{Tal Alexander Meeting 2019,}%meeting focusing on {\it Extreme Mass Ratio Inspirals}
     August 25 -- September 5, 2019 in Al\'ajar (Spain); {\bf invited blackboard presentation} on the dynamics of extreme mass ratio inspirals.

    \item \textsc{1$^{\rm st}$ LISA Astrophysics Working Group Workshop}, December 12--14, 2018 in  Paris (France); 
 {\bf talk} {\it Can supernova kicks trigger EMRIs in the Galactic Centre?}

    \item \textsc{YAGN18 meeting}, October 29--31, 2018 in  Budapest (Hungary); {\bf invited talk} {\it Can supernova kicks trigger EMRIs in the Galactic Centre?}

    \item \textsc{MODEST-18 meeting}, June 25--29, 2018 in Santorini (Greece); {\bf talk} {\it  EMRIs Triggered by Supernova Kicks in the Galactic Center}.

    \item \textsc{GRAvitational-wave Science \& technology Symposium 2018,} March 1 -- 2, 2018  in Padova (Italy);  {\bf talk} {\it Star Cluster Disruption by a Super-massive Black Hole Binary}.


    \item \textsc{Astro-GR@Barcelona 2017,} meeting focusing on {\it Extreme Mass Ratio Inspirals}, October 16 -- 20, 2017 in Barcelona (Spain).

    \item \textsc{ MODEST-17 meeting},  September 18 -- 22, 2017 in Prague (Czech Republic):
\vspace{-10pt}

    \begin{itemize}[leftmargin=5mm]
    \setlength\itemsep{-2pt}
        \item {\bf talk} {\it Star Cluster Disruption by a Supermassive Black Hole Binary}; 
        \item {\bf poster presentation} {\it Supernova Kicks and Dynamics of Compact Remnants in the Galactic Centre}; 
        \item {\bf chair}  of the {\it Galactic Centre / Nuclear Star Clusters} session.
    
    \end{itemize}
    \vspace{-8pt}

    \item \textsc{Lemaitre Workshop}  on Black Holes, Gravitational Waves and Spacetime Singularities, May 9 -- 12, 2017, Vatican Observatory, Albano Laziale (Italy); {\bf talk} {\it Massive Black Hole Binaries and their Hosts: a common journey towards Gravitational Waves}.

    \item \textsc{Cosmic-Lab – MODEST-16 meeting},  April 18 -- 22, 2016 in Bologna (Italy); {\bf poster presentation} {\it Dynamics of Supernova remnants in the Galactic Center}.

    \item \textsc{MODEST-15s meeting}, December 7--11, 2015 in Kobe (Japan);  {\bf talk} {\it Dynamics of Massive Black Hole Binaries in Galactic Merger Remnants}.

\end{etaremune}


\subsection*{Accepted computational proposals}

\begin{description}

    \item[\normalfont 2016:] \textbf{PI} of the proposal\textit{ Evolution of Massive Black Hole Binaries in their host systems} at CINECA, {\bf 99k CPU hours} awarded for $N$-body simulations on the GALILEO IBM cluster.

    \item[ \normalfont  2015:] \textbf{PI} of the proposal \textit{Evolution of Massive Black Hole Binaries} at CINECA, {\bf 99k CPU hours} awarded for $N$-body simulations on the GALILEO IBM cluster.
\end{description}

\subsection*{IT experience}

\begin{itemize}

    \item\textit{Operating Systems}: Linux/Unix, Windows

    \item\textit{Programming Languages}: {\bf C } (advanced), C++ (good), python (good), bash (good)
    \item \textit{Codes}: HiGPUs, HiGPUs-R, phiGRAPE, phiGPU, RAMSES, gasoline2

    \item\textit{Science-related softwares/packages}: ROOT, IRAF, SPLASH; python: Numpy, Scipy, yt, pynbody

    \item\textit{Plotting packages}: gnuplot, Supermongo, SPLASH; python: Matplotlib and Pyplot 
    
    \item\textit{Job Schedulers}: PBS, Slurm
    
    \item\textbf{Experience in GPU computing and parallel computing} (CUDA, OpenCL; openMP, MPI)
    
    \item Experience in code optimization and high performance computing 

\end{itemize}

\subsection*{Memberships}

\begin{itemize}
%\setlength\itemsep{-2pt}

    \item Full member of the {Astrophysics Working Group} of the {\bf LISA Consortium} since 2018; member of the LISA Early Career Scientist Group (LECS) since 2019.

    \item Co-I of the White Book for the scientific cases of  \textbf{MAORY}, the multi-conjugate adaptive optics module of the European Extremely Large Telescope.
    
\end{itemize}


\subsection*{Referee duties}
Reviewer for the {\it Monthly Notices of the Royal Astronomical Society} (MNRAS) since 2018.

\subsection*{Advanced courses and schools (highlights)}

\begin{itemize}
%\setlength\itemsep{-2pt}

    \item \href{http://ihpcss18.it4i.cz/}{\textit{International HPC Summer School 2018,}} {\bf highly competitive admission} (80 participants selected among $\sim$600 applicants); July 8--13, 2018 in Ostrava (Czech Republic)

    %\item Course { {\it Machine Learning for Astronomy}}, held for the PhD School in Astronomy at the University of Padova by Dr M. Pasquato, February 15--22, 2018 
    
    \item 11$^{\rm th}$ TRR33 Winter School in Cosmology, {\it Theory for Observers \& Observations for Theorists}, December 10--16, 2017, Passo del Tonale (Italy); { poster presentation} {\it Supernova Kicks and Dynamics of Compact Remnants in the Galactic Centre}, \textbf{  first prize for the best poster of the school} 
    
    %\item Course {\it Dynamics of stars and black holes in dense stellar systems} held for the PhD School in Astronomy at the University of Padova by Prof M. Mapelli, June 2017
    
    %\item {\it KROME computational school}, September 19--21, 2016 in Arcetri, Florence

    \item CINECA school \textit{{ 25$^{\rm th}$ Summer School on Parallel Computing}}, June 6--17, 2016 in Casalecchio di Reno, Bologna (Italy)

    %\item CINECA course \textit{{ Python for computational science}}, February 9--11, 2016 in Segrate, Milan
    
    %\item Course {\it N-body techniques for astrophysics} held for the PhD School in Astronomy at the University of Padova by Prof M. Mapelli, November 2015.
    
\end{itemize}

\subsection*{Outreach activities}

\begin{itemize}
%\setlength\itemsep{-4pt}
     \item On-line outreach seminar for the Italian \textit{AstronomiAmo} Association, 2020;

    \item Public outreach seminar at the Auditorium of Varano Borghi (Italy), 2020;
    
    \item  Public outreach talk for the Livio Gratton Prize  in Frascati (Italy), 2019;%\\
    %Title: \textit{Quando due giganti si incontrano: una storia sui buchi neri supermassicci} %(The rendez-vous of giants: a supermassive black hole story)

    \item Outreach seminar at the Padova Planetarium, 2018;%\\
    %Title: \textit{A caccia di buchi neri con le onde gravitazionali}% (Hunting black holes with gravitational waves)

    \item Member of the committee for the organization of \href{https://pintofscience.com/}{\it Pint of Science} 2018 in Padova (Italy);
    
    \item \href{http://www.venetonight.it/}{\it Notte dei Ricercatori} (The night of researchers) 2017, public outreach events in Padova  (Italy);
    
    %\item[2017:] INAF press release for the Magini Prize 2016: %\href{http://www.media.inaf.it/2017/05/29/premio-magini-arcetri/}{www.media.inaf.it/2017/05/29/premio-magini-arcetri/}
    
    \item Support for the monthly public star gazing events held at the University of Surrey telescope, 2016 -- 2017.

\end{itemize}



\subsection*{Institutional responsibilities and Service}

\begin{description}

    \item[ \normalfont 2020:] \textbf{Coordinator for the drafting} of Chapter 2 (Massive black holes) of the \textbf{LISA Astrophysics Working Group White Paper}

    \item[ \normalfont 2019 -- 2020:] Organizer of the Seminars for the Institute of Computational Science, U. of Z\"urich
    
    \item[ \normalfont 2019:] Main organizer of a series of  events  for the \textbf{promotion of Diversity in Science}, U. of Z\"urich; main speaker and moderator in one meeting (\href{https://padlet.com/eli_bortolas/39e0898pr0uu}{https://padlet.com/eli\_bortolas/39e0898pr0uu}).

    \item[ \normalfont 2013 -- 2015:] Student representative of the Astrophysics and Space Physics master students, U. of Milano-Bicocca.
    
\end{description}


\subsection*{Language knowledge}
Italian (mother tongue), English (fluent), French (basic).%, German (basic).
